This monograph presents some constructive ideas about epistemology, about how we can acquire and apply knowledge.
Calling these ideas constructive reflects a purpose distinct from descriptive epistemology but falling short of prescriptive epistemology, offering an epistemological synthesis and some analysis of its merits for consideration as a practical way of advancing the acquisition, representation and application of knowledge to the benefit of mankind.

Three concerns motivate me in this matter.

The first is esoteric and technical, and concerns the exploitation of recent developments in mathematical or symbolic logic.
These suggest that some of the formal logical systems devised within the last 100 years may be suitable for the representation of knowledge, and that this kind of knowledge can be regarded as foundational and other kinds then built upon that foundation.
A particular concern in this matter is the adoption of such methods of representation provides a means whereby \emph{Artificial Intelligence} can be made reliable within the scope of the methods, which include all manifestations of deductive reasoning.

My second area of interest is in the nature of scientific truth, the status of scientific authority and the idea of scientific \emph{proof}.
This falls into two parts.
The first part concerns the application of logic to the construction of abstract models of the physical world, and the ways in which such models may be evaluated.
The second part concerns the institutions of science and engineering, looking at these from a sceptical point of view and questioning how such scepticisms might be allayed.

My final area of concern is with morality and politics, with the tribal surrender of rationality to ideology and how to limit the social damage which may result.

Of these three areas of concern, the first is one in which I have some straightforward (if somewhat technical) observations to make and am, insofar as I have one, on home territory.
Regrettably this may also be the least important and urgent of the three to which I might make a contribution.

In the second two I move progressively further away from any claim to competence, but feel compelled to seek insight nonetheless by the perceived importance of these areas in contemporary society.

