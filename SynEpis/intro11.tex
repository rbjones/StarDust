

\ignore{
There are many ways in which the early decades of the twenty first century are remarkable and portentous.
Much of it relates to advances in science and technology.

Epistemology is concerned with knowledge, which is related to intelligence, and to some of the technologies at present thought to be approaching or traversing a point of inflection.

Intelligence itself has been central to speculation over a few decades about an approaching technological singularity resulting in the advance of ``artificial intelligence'' to and beyond the level of  human intelligence, promising radical acceleration of the rate at which knowledge is gathered and inviting further innovation in the management of large volumes of shared knowledge.

\section{The Technological Singularity}

\cite{good1965,vinge1993,kurzweil2005}
}%ignore

Knowledge is a physical phenomeon in which the structure of some physical medium represents information about \emph{something else},
some other part or aspect of the physical world, or perhaps about abstractions.
Often the correspondence is secured by a causal relationship between the two, the state of the representation being causally dependent on the state of the thing represented.

The forms in which knowledge has manifest during the evolution of life on earth are varied, and have evolved along with life itself.
We may think of this evolution as falling into three principle phases.

In the first phase, we may think of evolution as accumulating knowleddge of how to build organisms capable of surviving and reproducing in various environmental niches which is encoded in the genomes of the organic species.
These organisms exhibit knowledge of how to thrive and procreate.
Insofar as the organisms are able to sense their environment and respond appropriately, we may see, in the sensory-motor pathways which which effect those capabilities, representations of the relevant aspects of the environment, which ultimately effect the necessary response.

\section{}

An important distinction between the kinds of knowledge observable in this domain is that between knowing \emph{how} and knowing \emph{that}, of which the former has dominated for most of the history of life on earth.
Whenever an organism has a capability which contributes to its ability to survice and reproduce, we may say that it \emph{knows how} to exercise that capability, and that knowledge will be found to have one or more physical realisations.
In very primitive organisms, the capability may be genetically endowed, and the knowledge may then be said to lie both in the genetic codes which control the development of the relevant physical characteristics, and in those physical structures which effect the capability.
In more advanced organisms, the capability may be learned, and its embodiment might then be seen in the neural structures, neurons and their synaptic connections, which control the relevant behaviours.

The physical realisation of such an adaptive response may itself be processing knowledge, for the history involves increasingly elaborate ways of processing and responding to environmental stimuli, at each stage of which information about the sensed environment or about how an appropriate response can be marshalled is involved, all of which may be thought to constitute knowledge represented in some way by physical processes within the organism.

In the latter part of the evolutionary history of our ecosphere, knowling how to survive and replicate is supplemented and further enabled by knowing \emph{that} certain propositions are true, and declarative knowledge begins to supplement the procedural knowledge hitherto dominant.
In this description, it seem that declarative knowledge depends upon language, not merely to identify the content of the knowledge, but also as its representation.
This might suggest that declarative knowledge is a purely linguistic phenomenon, and is coeval with language in human beings, perhaps even co-evolved with linguistic competence.

It seems however, that the emergence of declarative knowledge was a more gradual and confused business, for as soon as we try to give a clear characterisation of declarative knowledge we find that most language has and still falls well short of the ideal which such a characterisation delineates.
