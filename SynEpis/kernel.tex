The word `philosophy' derives from a greek word meaning `love of knowledge', and in the time of Aristotle embraced what were then known as the sciences.
The sciences as we now know them are no longer considered part of philosophy.
One alternative conception of philosophy has been that of providing an intellectual context or foundation in which the sciences might be conducted.
This is related to Aristotle's undertaking in the volume which became known as his \emph{metaphysics}, describing it as `first philosophy', or `the study of being \emph{qua} being'.
By that he meant the study of those aspects of all that exists (substance) before considering the particular characteristics which qualify something to be addressed under one of the sciences, i.e. those characteristics which are \emph{prior} to or more general than science in some way.

The term 'first philosophy' has been used more recently by philosophers deprecating approaches to philosophy predicated upon philosophy having something to say which is relevant to and prior to science.
Nevertheless, historically, in the Western tradition which has descended from the philosophy of Classical Greece, once philosophy was distinct from science, it has been common for the theory of knowledge, `epistemology' to be considered its home ground.

Other contenders for that pivotal status in modern philosophy have been:
\begin{itemize}
\item \emph{Metaphysics}, both because of its status in Aristotle's philosophy and because this is a pursuit which can be set apart from empirical science and studied in its own right by deductive rather than empirical methods.
  
  Perhaps because the ambitions of metaphysicians have often greatly outstripped the capabilities of their tools (pure reason), the idea that philosophers are not concerned with the concrete world so much as with the analysis of language, has resulted in periods in which:
  \item \emph{philosophy of language} has seemed to be philosopher's most central concern.
    Though what philosophers call `ordinary' language is substantially an accident the study demands empirical methods, and perhaps even an unfortunate accident which for the sake of rigour in both philosophy and science we should discard in favour of the logical notations which emerged after the rigorisation of mathematical analysis in the 19th Century, leading to the conception that philosophy is most fundamentally concerned with
  \item \emph{logical analysis}
    which sometimes means little different to the analysis of ordinary language
\end{itemize}

I am myself with those who regard philosophy's most vital role in making possible the accumulation of knowledge and its application to the benefit of mankind by sometimes elaborate predictions about the behaviour of the material world mediated by complex deductive reasoning from our shared body of knowledge.
Alongside that, I see the practical effectiveness of foundational thinking in elaborating and applying such methods.

The philosophical kernel which I introduce here is \emph{foundational} in seeking to provide a general concept of, and abstract representation for, knowledge suitable for use in the sciences and elsewhere, a very general way in which knowledge can be represented which is semantically precise and coherent, and the means to engage in extended deductive reasoning in the application of scientific knowledge to the furtherance of human purposes.

The foundations upon which this rests are \emph{logical}, supporting reasoning in any domain suitable for coherent deduction (arguably,  any coherently well-definable domain) upon the general notion of \emph{logical truth} (as here defined).


