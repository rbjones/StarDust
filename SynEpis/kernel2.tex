The proposed system for the representation of knowledge is \emph{foundational}, it is a system to which other ways of representing knowledge are reducible in some sense (to be described).
It is therefore necessary in its articulation to speak in general of knowledge representation systems and of the idea of reduction relative to which the proposed system may be seen as universal in a broad class of such systems.

Though the proposal is \emph{epistemological}, its description depends upon \emph{metaphysics} (particularly, \emph{ontology}), \emph{philosophy of language}, and \emph{logic}, those four aspects of philosophy combining to articulate the most fundamemtal parts of the system.

\emph{Foundationalism} has been regarded as a failed doctrine by many philosophers for a good while, but this may well be because they consider only a straw man, in which foundational theses always make absolute claims about the most fundamental parts of their systems.
This I will not do, I aknowledge that neither in relation to meaning nor verification can one ever have absolute precision or certainty.
One can however, in certain domains which I will present as of particular importance, come closer to those ideals than any practical purpose demands.
Foundations accomplish that end, but they do so from a particular philosophical perspective, so technical adequacy will not necessarily by universally convincing.

The foundational problem in relation to both meaning and truth is how to terminate infinite regress in definition or justification, and this may be presented as a choice between finding a foundation which is self-evidently clear and conclusive, or rendering the foundation precise through the use of language ultimately to be defined in terms of it.
In this proposal, this choice is rejected in favour of doing both.

The foundational enterprise can be appreciated as the threading of language through the eye of a needle.
The whole of complex languages are to be defined in terms of very simple primitives, which are definable using a tiny part of the languages which can then be constructed upon them.

\section{Epistemological Abstraction}

Epistemology has often been in significant measure influenced by the world around us, and the ways in which human beings are built and acquire and deploy knowledge of themselves and the world around them.
It may also be influenced by or intimately concerned with the language with which we talk about knowledge, not least the meaning of the word ``know''.

The approach here, which we may think of as an approach to `abstract epistemology', seeks to minimise the extent to which epistemology reflects such earthly or anthropomorphic influnces.
How could such an epistmology arise, what would be the purpose of attempting to construct such an abstract epistemology, and how could it possibly succeed?

This moment in the evolution of intelligence provides a context in which this might be understood.
We stand, as I write, at a point at which many of the hallmarks of intelligence in humans are now to be seen in computational artifacts.
We are also venturing into synthetic biology which may transform the evolution of biological intelligent systems, and the ambition to send intelligent systems across the solar system and into the star systems beyond is on the ascendent.
It is likely that the promulgation of intelligence across the galaxy will ultimately be predominantly of non-biological intelligence, and that even that central core where biological life has penetrated will be populated by species well advanced beyond homo sapiens, growing progressively more distant from earth and homo sapiens.
These are among the motivations to consider epistemology in ways which stand back both from human language about knowledge and human ways of acquiring knowledge.

Pure mathematics provides examples of structures, knowledge of which will surely be universal among intelligence wherever it is found.
The natural numbers, those numbers which we use for counting discrete entities, are a simple example.
Abstraction to is the business of pure mathematics, in contact with an alien civilisation, it may be the mathematicians who would best suceed in communicating with their alien counterparts.

\section{Foundational Metaphysics}

The abstract foundation for epistemology here envisaged is a story couched in terms of abstract entities.
So what are they, and what abstract entities are there?

The distinction between abstract and concrete is not completely clean, since we can construct abstract entities with concrete constituents.
We are here concerned with purely abstract entities, and adopt a conventionalist position in relation to such entities.
The question what abstract entities there are is therefore to be determined by context, an aspect of context which might (or might not) be fixed by the language.
If it were meaningful to speak of there being absolute truths about what abstract entities exists, it would not impact this position, for the utility of \emph{chosing} a domain of abstract entities for the purpose of constructing an abstract model or for developing a mathematical theory is not dependent on what abstract entities do or do not `really exist'.

Concrete ontology is not far removed from this conventional stance, though the evaluation criteria which is makes sense to apply to concrete ontologies are more stringent.
Concrete ontology, we might expect, is primarily of utility in constructing models of the physical world, and may be subject to similar criteria.
Though philosophers have debated the validity of inductive reasoning to establish the truth of empirical generalisations, and have proposed alternatives such as continuous search for falsification, it is clear that the utility of a model of the empirical world may persist in the face of good evidence that it is literally false.
Newtons theories of motion and gravitation are the clearest examples, where accepted as false they are nevertheless more widely used than the theories which displaced them and are still thought to be `true'.

\section{Abstract Languages}


