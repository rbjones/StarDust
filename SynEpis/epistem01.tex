
\section{Evolution and Epistemology}

To give a first account of the way in which evolution and epistemology are entwined in this narrative, I must provide characterisations of those two concepts, both of which I take in a broad sense sympathetic to the ideas whose exposition they abet.

\subsection{What is Evolution?}

The term evolution is used here for any process of progressive or incremental change whose long term effects are realised through some kind of differential proliferation.
`Differential prolifearation' occurs when certain kinds of entities proliferate (multiply and/or disperse, replication admitting some degree of variation) at rates which vary according to their particular characteristics.
The basic principle here is that those kinds of entity which proliferate most profusely within some environmental niche will come to numerically dominate in that niche.
Those small changes which improve proliferation are considered `adaptive', and over time these small changes may yield the major transformations which are seen (for example) in the evolution of species.

However, this conception of evolution lacks some of the characteristics usually seen in the evolution of species.
In the classical conception, the variations which are essential to evolutionary progress are supposed random, and proliferation is moderated by `natural selection' which determines which variants prove most prolific.
These characteristics may be helpful in advertising a natural process of evolution which does not depend on divine intervention, and may be accurate descriptions of biological evolution, but are not found in all the kinds of evolution which are of interest here.
Prebiotic evolution, cultural evolution, or the kinds of evolution which may yet emerge as artificial intelligence and synthetic biology mature to dominate evolution in the future.

\subsection{What is Epistemology?}

Though talking about knowledge is nearly as old as language itself, epistemology, the philosophical theory of knowledge, probably begins with Plato, who spoke of knowledge as \emph{justified true belief}.
Insofar as we may infer the meaning of the term from its usage, which is diverse, this is a narrow characterisation.
It is psychologistic, anthropocentric, and addresses only the kind of rigorous knowledge we associate with science, to which philosophy often aspires.

The kind of epistemology in which this monograph engages I call \emph{synthetic}, and it yeilds a synthesis or construction of a conception of knowledge primarily defined by the manner in which knowledge is \emph{represented}, diversifying the metrics evaluating supposed knowledge away from `justifiction' to admit criteria appropriate to the full diversity of knowledge, including, for example, procedural  knowledge, or \emph{knowing how}.

The epistemology here is \emph{synthetic}, which can be understood by contrast with three other approaches to epistmology which are:
\begin{itemize}
\item[analytic]

  This is the kind of philosophy which might be most appropriate to the analytic tradition in philosophy, particularly its most recent manifestations in the early to mid twentieth century.
  It takes language \emph{as it is}, perhaps even with an emphasis on \emph{ordinary language} and enquires what terms like `know' and `knowledge' mean in that established usage.
  It may be noted that in talking about `analysis' here, we are primarily speaking of analysis of language rather than logical analysis, and that this process properly yeilds synthetic truths about a natural phenomenon rather than logical truths.
\item[prescriptive]
  epistemology becomes prescriptive when it is concerned to determine how the relevant language \emph{should} be used rather than how it \emph{is} used.
  \emph{natural}
  Natural epistemology, or epistemology naturalised, is a kind of epistemology beginning later in the twentieth century, initiated by W.V.Quine \cite{quine1969epistemology}, in which knowledge is considered a natural phenomenon and should be persued by the methods of empirical science.
\end{itemize}
