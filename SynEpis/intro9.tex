This monograph is dedicated to the presentation of a well established logical system as providing a universal representation system for knowledge.
In the course of explaining the rationale for these ideas, the logical kernel of a foundational philosophical system, within which the ideas can be understood, will be outlined.

In this introductory chapter, some historical background will be introduced, beginning with Isaiah Berlin's ideas about ``the Western Tradition'', (as it stood in the Enlightenment) contrasted with Hume's skepticism, reaching back to the origins of that tradition in Plato and Aristotle before considering the origins of declarative language and that thread of its evolution which lead to formal logical foundations for mathematics, their transformation for application in formal verifiation of digital electronics and this repurposing as a substrate for linguistically pluralistic broad spectrum knowledge representation system.

\section{Berlin on the Western Tradition}

In his book on \emph{The Roots of Romanticism}\cite{berlinRR} Isaiah Berlin talks about the  ``three legs upon which the whole Western tradition rested'', before the enlightenment, as follows:
\begin{enumerate}
  \item All genuine questions can be answered.

    In principle, by someone.  Perhaps only God.
\item  The answers are knowable.
\item All the answers are compatible (with each other).
  It is a logical truth, Berlin says, that one true proposition cannot contradict another.
\end{enumerate}

and then, the extra twist added by the Enlightenment:
\begin{quotation}
That the knowledge is not to be obtained by revelation, tradition, dogma, introspection..., only by the correct use of reason, deductive or inductive as appropriate to the subject matter.

This extends not only to the mathematical and natural sciences, but to all other matters including ethics, aesthetics and politics.
\end{quotation}
and... that virtue is knowledge.

This is a simple description, not of reality, but of an unattainable ideal, which was to be repudiated by romanticism.

There are two separate kinds of issue which I will raise about this idea.
The first is whether this supposed Western Tradition has a factual basis and utility, the second is whether it is a good characterisation of Enlightenment thought.
Notwithstanding the reservations which I will present, I consider that there is a core to this which is important, and that the utility of declarative language in advancing human prosperity depends both upon such language approximating a similar ideal and on the evolution of some specialised parts of language (for example the languages of mathematics and quantitative science).
This monograph may be seen as an advocacy for pressing forward that evolution exploiting the growing capability of artificial intelligence.

Lets water this down a little.
We'll drop the omniscience 

\begin{centering}
*****************************
\end{centering}
  
Though declarative language does not uniformly comply with this ideal, the mere approximation has been crucial to the progression of humanity from mere subsistence to prosperity and increasing mastery over his environment, and in those areas where elaborate deductive reasoning is beneficial, language has evolved to greater precision and methods of reasoning have become more systematic and rigorous.

It will not be expected that all questions have an answer, for it is convenient sometimes to work with entities for which we have only incomplete descriptions, but it is an aspiration that any question definite enough to be amenable to deductive reasoning, either in its establishment or its application, can be accommodated within the synthesis.

There are now strong reasons to doubt that the answer to any properly formulated question can be discovered and established.
In many aspects of the proposed synthesis, absolutes are known to be unrealisable, and it is more important to be confident in the answers which do come than for such answers to be always forthcoming.

That all the answers be compatible is possibly the most crucial requirement in a system intended for extended deductive elaboration, for in default of coherence, no result can be trusted.

Clearly this is an idealisation

\section{The Philosophy of David Hume}

\section{Logical Foundations}

For reasons which have been extensively discussed, there is no single language and logic which can be universal, either as a foundation for meaning or truth, and so the system discussed here is a family of  

The central argument falls into two parts.
First the claim is made that the logical system is universal foundation for logical truth, that logical truth in any well-defined language can be reduced to logial truth in this system.
The system

Then it is argued that logical truths in any well-defined language is reducible to truth in the proposed logical foundation system.
Then 


It is not primarily concerned with presenting the technical detail of this system, which has been done elsewhere, but rather with arguments explaining and supporting its broader significance and appliocability.
than is widely recognised, and in answering certain sceptical arguments which have played a prominent role in analytic philosophy during the last century.

The system 

It contains a certain amount of fundamental philosophical thinking, presenting the logical system as \emph{foundational} and articulating a philosophical kernel around it.
