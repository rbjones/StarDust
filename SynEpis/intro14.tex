This epistemological synthesis is the first part in a progression which might ultimately advance from epistemology though architectural design, API and protocol specifications to code.
The boundary between epistemology and the earliest parts of system architecture is fuzzy, and the same subject matters can be addressed both from a philosophical and an engineering perspective.
This monograph focuses on the former, but is motivated by the latter.

The central purpose is to suggest that a single underlying abstract logical system suffices for the representation of all declarative knowledge, and that the forthcoming spread of intelligence through the solar system and across the galaxy, can best be accomplished with the aid of a distributed knowledge base using that single abstract representation (but diverse concrete representations).

The proposed logical system, closely based on the logic devised for and used in the Cambridge HOL Interactive Theorem Prover\cite{gordon1993}, is argued to be, in important theoretical and practical senses, universal for the representation of declarative knowledge.
It is offered as a foundation for abstract semantics in general, not merely a logical foundation for mathematics (for which its predecessors were intended).

Why do we need a 'foundation' for 'abstract semantics', and what are they?
These are just some of the philosophical questions which I hope to clarify in this monograph.
We cannot be certain of the truth of a declarative sentence if its meaning is unclear, and that same difficulty will arise for any language in which we might seek its clarification.
This is the problem of semantic regress, and the role of a semantic foundation is to terminate that regress.

The clarification of and justification for these claims forms an important part of the purpose and substance of this monograph, alongside some arguments intended to counter the scepticism about semantics to which philosophers are prone, particularly in relation to problem of semantic regress and that of regress in justification.

In these first introductory words on these ideas I will give two different perspectives on this same proposal.
But first some observations about desiderata.

The development of science and technology, upon which the prosperity of humanity has been built, is based on scientific methods in which declarative knowledge and deductive reasoning are central.
The systematic and extended use of deductive reasoning dates from the beginnings of mathematics as a theoretical discipline in ancient Greece, which resulted in the articulation of the axiomatic method and the elements of mathematics by Euclid which has remained influential to the present day.
Axiomatic mathematics was influential among Greek philosophers because they sought to understand the cosmos through reason, and by contrast with the advances in mathematics, reason proved impotent to secure consensus on truth in those broader domains.

Some Greek Philosophers nevertheless felt that Science should be a deductive discipline, and this is most explicitly presented in Aristotle's \emph{Organon}, his works on logic, in which the idea of \emph{demonstrative science} is presented and supported.
That early conception of science fell well short of modern scientific methods, and Aristotelian science was overtaken by one in which the deductive application of scientific principles was considered 

\paragraph{Declarative Language}

Declarative language, that expressed by indicative sentences in ordinary language, or in formal notations designed for similar purposes.
These are sentences which have truth values in appropriate contexts.
The context in which such a sentence is expressed determines a certain range of possibilities, and the sentence will usually vary in truth value according as to which of those possibilities obtains.
The truth of such an indicative sentence narrows down the range of possibilities and thereby conveys information about how the world (or other subject matter) is.

\section{Knowledge}

The classical definition of knowledge first appearing in the works of Plato, is that knowledge is \emph{justifiable true belief}.
However, this monograph addresses a broader conception of knowledge which is likely to be more convenient for the management of large bodies of knowledge.

This is not a claim about what knowledge \emph{is}, it is merely a description of the particular usage adopted in this monograph.
Relative to the traditional perspective, it differs by being neither psychological nor homocentric.
Knowledge is not exclusively mental, but predominantly physical.

Knowledge is a physical phenomenon in which some part of the physical universe contains information about some other thing, often another part of that universe.
Usually that it does so derives from its origin in a causal relationship between the two, though this may be very indirect and tenuous.

Thus, we acquire through our senses knowledge of our environment, because there is a causal chain which begins with various features of the environment, acts through the media of our senses (light, sound, physical pressure) and is conveyed to, processed and stored in our brains by the neural networks which form our central nervous systems.

More obliquely, on the basis of such observations of the universe around us, scientists formulate and test hypotheses about that universe, which if shown to be accurate become part of the body of scientific knowledge recorded in scientific journals and held in a variety of storage media in automated information systems.

The purpose of encoding information in this way is to enable people to live their lives better as a result of planning how to achieve their objectives in a way which takes into account the likely outcome of such plans.
Knowledge about how the world \emph{is} is crucial to this, as is knowledge of how the world would come to be if our plans were put into effect.

For knowledge to be useful in this way the relationship between the representation and the represented must be understood both by the people creating the knowledge or be consistent with the causal chain which effected it, and by those seeking to take advantage of the knowledge.
Indeed useful or not, one cannot reasonably be said to possess knowledge if you are merely in possession of the representation but not its meaning, or what we call more specifically, its semantics (to distinguish it from aspects of meaning which might involve matters more related to its significance than its content).

Semantics then, is an essential aspect of the representation of knowledge, and it is something which attaches not to the specific content of a single representative of knowledge, but to a general method of encoding a certain kind of knowledge of which it is an instance.
Such general schemes include what we call languages, but may include coding schemes which we might not think of in that way, or might only be figuratively describing as knowledge.
The coding of the structure of proteins in the DNA of the chromosomes of living organisms is an example of such an encoding.
We might also consider a photograph to be knowledge of its subject, but might struggle to identify the language in which it is expressed.

In this, the representation of knowledge may be thought to encompass not only language, but also the more diverse phenomena of symbiosis.

\section{Declarative Language and its Universal Foundations}

Declarative knowledge, as that term is to be used here, is more narrowly scoped, and is more particularly relevant to the process of exploiting general knowledge for the purpose of planning and engineering.
This process of applying general knowledge... 

\cite{dretske1981}


%\section{Set Theoretic Foundations}

%\section{Type Theoretic Foundations}

