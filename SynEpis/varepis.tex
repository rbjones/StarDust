Knowledge is power, it is sometimes said.
The power, for example, which propelled humanity from hunter gatherers to farmers, built cities and civilisations, and enabled the industrial revolution, information technology and ``artificial intelligence''.

Much of the knowledge behind that trajectory, was \emph{propositional} knowledge, expressed in \emph{declarative language}, which could not have been instrumental without precision of language and objectivity of truth, gradually refined over millenia.

The body of scientific and technological knowledge instrumental in that process was a small part of diverse cultures made possible by language, probably co-eval with \emph{homo sapiens}.
The power of culture to promoting the welfare of the society in which it evolves depends both upon uniformity in some and diversity in other aspects.
Precision of language and objectivity of truth do not suffice to realise that uniformity, but social behaviour, of which a hallmark is uniformity of behaviour, dates back to the very earliest stages in the evolution of life on earth, and continues to exert a more profound effect on human behaviour than the fragile rationality enabled by large brains and declarative language.

Epistmology, the philosophical study of knowledge, in this context may be undestood as consisting in the attempt either to understand or to influence or contribute to the advancement of the ways in which these conflicting influences on the continuing rgrowth are reconciled.

