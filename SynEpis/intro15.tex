In this monograph I engage in certain philosophical preliminaries to the design and implementation of software for the management of large widely distributed repositories of declarative knowledge.

In this introduction I aim to describe the main points to be addressed, the breakdown material into chapters, and give an account in some detail of how the chapters will approach their task.

The core idea with which I begin, is that all declarative knowledge may be understood as providing abstract models of reality which support reasoning about that reality, sometimes enabling the planning of courses of action or design of artefacts.
In their application to reality such abstract models will be given concrete reality, but in the implementation of software support an underlying abstract representation suffices.

Humans use diverse languages and notations to express and work with declarative knowledge.
This and other factors introduce risk of incoherence, which will vitiate the results of any extensive reasoning.
These risks could be mitigated by the adoption of a single underlying abstract logical system, which is what I propose in this monograph.
To support that case, I will offer a specific system which includes a universal abstract syntax into which the abstract syntax of any other finitary declarative language can be mapped, and a rich semantics, enabling the semantics of other languages to be attached to that syntactically embedded language.

To support these claims of universality I will provide a description of two such universal systems in a manner intended to expose those merits.
Of the two systems which will be considered, one is presented as idea for the presentation of universal abstract semantics, and the second as having that same feature in a more complex form, as well as a better structure for interpreting practical languages.

\section{Chapter 2 - Some History}

The presentation will begin in chapter 2 with a history showing how the key features of these systems have evolved.
The history is oriented to exposing the development of the semantics, with a particular interest in ontology.
In it the origins are traced of two logical systems.
The first of these is the first order set theory whose axiomatic presentation is known as ZFC, Zermelo-Fraenkel with choice.
The second is the variant of the simple theory of types due to Alonzo Church\cite{church1940}, which is discussed both in the form originally published by Church and then as adapted by Michael Gordon for use in the formal verification of digital electronics.

The history comes in three parts.
The first part very briefly sketches the early history of logic and the extended use of deduction in mathematics over 2 millenia from Aristotle to Dedekind, at which point mathematics is well advanced in addressing a foundational crisis.
That crisis caused clarification of multiple layers of mathematical concepts, eventually reaching down to the bedrock of what we might now call \emph{logical foundations for mathematics}, with Dedekind defining the condition for a collection or set to be infinite in number\cite{dedekind1847}, thus for the first time making the idea of a complete infinity respectable and advancing the status of sets in the clarification of foundational issues.

The second part of the history looks more closely at an astonishingly fertile century in the story of logic and the foundations of mathematics, firmly establishing in the process the new discipline of \emph{mathematical logic}.
By the close of the century which started in 1847 with Dedekind's paper on the definition of infinite collections, two logical systems were in place which are central to the story, the set theory ZFC formalising the theory of the cumulative hierarchy of well-founded sets described by Van Neumann\cite{neumann1930}, and Church's variant of the Simple Theory of Types (STT)\cite{church1930}.


In the third part of the history we migrate into Computer Science to describe the further developments to Church's STT which were adopted by Mike Gordon for use in hardware verification.
The documentation for the resulting system is briefly discussed in Appendix \ref{HOLspecs}, but the final part of the history gives a more concise account of the nature and importance of the innovations.

There are a number of themes which the history is intended o bring out, which together contribute the explanation and justification of the idea that both first order set theory and Gordon's HOL are suitable bases for universal foundations systems for abstract semantics, and thence for a representation of declarative knowledge.

A dominant focus is on the development of semantics, and within that the significance of ontology.
Aother line of development concerns the concepts around logical systems, the kinds of system under consideration and the relationships between these different kinds of system, leading up to a conception of semantic reduction which creates an ordering by expressiveness of the systems involved and hence invites the question whether there could be any systems which are maximal in expressiveness, and hence might be termed universal semantic foundations.
The kinds of distinction here for example, are that between:
\begin{itemize}
\item A `logic' in the sense of first order logic, which we might characterise as a language with semantics and inference rules, which can be extended axiomatically to address arbitrary  mathematical theories.

\item A mathematical foundation system of the kind which would fulfill the role of logic in Frege's ``mathematics = logic + definitions'' by supporting the development of mathematics using only the foundation system and definitions of the relevant concepts.
First order set theory, Russell's Theory of Types, and Church's STT are all mathematical foundations systems of this kind.

\item
These kinds of system do not admit universality of the kind here sought and the reasons for this will be discussed, in peparation for the identification of a distinct kind of system which I will be calling a semantic foundation system, which both ZFC, and Gordon's HOL can form a basis for, given some small elaborations of their semantics and axioms.

\end{itemize}

\section{Chapter 3 - Semantic Foundation Systems}

This chapter is devoted to the definition of:
\begin{itemize}
\item Inductive finitary abstract syntax
\item Universal abstract syntax
\item Semantic Foundation Systems
\item Semantic reduction
\item Universal semantic foundation systems
\end{itemize}

and argues that Gordon's HOL provides both a universal abstract syntax and a universal semantic foundation system.

The case for universality is supported by other examples and some systematic ways of constructing a variety of such systems.

\ignore{\cite{scott_strachey_1971}}
